%%%%%%%%%%%%%%%%%%%%%%%%%%%%%%%%%%%%%%%%%%%%%%%%%%%%%%%%%%%%%%%%%%%%
%% I, the copyright holder of this work, release this work into the
%% public domain. This applies worldwide. In some countries this may
%% not be legally possible; if so: I grant anyone the right to use
%% this work for any purpose, without any conditions, unless such
%% conditions are required by law.
%%%%%%%%%%%%%%%%%%%%%%%%%%%%%%%%%%%%%%%%%%%%%%%%%%%%%%%%%%%%%%%%%%%%

\documentclass[
  digital, %% This option enables the default options for the
           %% digital version of a document. Replace with `printed`
           %% to enable the default options for the printed version
           %% of a document.
  twoside, %% This option enables double-sided typesetting. Use at
           %% least 120 g/m² paper to prevent show-through. Replace
           %% with `oneside` to use one-sided typesetting; use only
           %% if you don’t have access to a double-sided printer,
           %% or if one-sided typesetting is a formal requirement
           %% at your faculty.
  table,   %% This option causes the coloring of tables. Replace
           %% with `notable` to restore plain LaTeX tables.
  nolof,     %% This option prints the List of Figures. Replace with
           %% `nolof` to hide the List of Figures.
  nolot,     %% This option prints the List of Tables. Replace with
           %% `nolot` to hide the List of Tables.
  %% More options are listed in the user guide at
  %% <http://mirrors.ctan.org/macros/latex/contrib/fithesis/guide/mu/fi.pdf>.
]{fithesis3}
%% The following section sets up the locales used in the thesis.
\usepackage[resetfonts]{cmap} %% We need to load the T2A font encoding
\usepackage[T1,T2A]{fontenc}  %% to use the Cyrillic fonts with Russian texts.
\usepackage[ruled,vlined,linesnumbered,noresetcount]{algorithm2e}
\usepackage[
  main=english, %% By using `czech` or `slovak` as the main locale
                %% instead of `english`, you can typeset the thesis
                %% in either Czech or Slovak, respectively.
  english, german, russian, czech, slovak %% The additional keys allow
]{babel}        %% foreign texts to be typeset as follows:
%%
%%   \begin{otherlanguage}{german}  ... \end{otherlanguage}
%%   \begin{otherlanguage}{russian} ... \end{otherlanguage}
%%   \begin{otherlanguage}{czech}   ... \end{otherlanguage}
%%   \begin{otherlanguage}{slovak}  ... \end{otherlanguage}
%%
%% For non-Latin scripts, it may be necessary to load additional
%% fonts:
\usepackage{paratype}
\def\textrussian#1{{\usefont{T2A}{PTSerif-TLF}{m}{rm}#1}}
%%
%% The following section sets up the metadata of the thesis.
\thesissetup{
    date          = \the\year/\the\month/\the\day,
    university    = mu,
    faculty       = fi,
    type          = bc,
    author        = Martin Geletka,
    gender        = m,
    advisor       = Mgr. Samuel Pastva,
    title         = {Implementation of Cylindrical Algebraic Sub-decomposition},
    TeXtitle      = {Implementation of Cylindrical Algebraic Sub-decomposition},
    keywords      = {Polynomial zeros, Real algebraic geometry, Polynomial projection, Computational algebra, Real algebraic geometry},
    TeXkeywords   = {Polynomial zeros, Real algebraic geometry, Polynomial projection, Computational algebra, Real algebraic geometry},
    abstract      = {The goal of my bachelor thesis is to study and implement the algorithm of Cylindrical Algebraic Sub-decomposition.  The algorithm is used as the primary tool to study of semi-algebraic sets, which have practical use in research of continuous and hybrid models. 
Need of this thesis rest in the actual state of available implementations. These implementations are outdated or only included in commercial tools as Maple.
 Because of that the final application will be available as an open source library and used for further research of semi-algebraic sets.},
    thanks        = {I would like to express my sincerest gratitude to my supervisor, Mgr.
Samuel Pastva., for his guidance and the lots of invaluable advice he gave me during the
composition of this thesis.},
    bib           = example.bib,
}
\usepackage{makeidx}      %% The `makeidx` package contains
\makeindex                %% helper commands for index typesetting.
%% These additional packages are used within the document:
\usepackage{paralist} %% Compact list environments
\usepackage{amsmath}  %% Mathematics
\usepackage{amsthm}
\usepackage{amsfonts}
\usepackage{url}      %% Hyperlinks
\usepackage{markdown} %% Lightweight markup
\usepackage{listings} %% Source code highlighting
\usepackage{amssymb}
\usepackage{chngcntr}

\lstset{
  basicstyle      = \ttfamily,%
  identifierstyle = \color{black},%
  keywordstyle    = \color{blue},%
  keywordstyle    = {[2]\color{cyan}},%
  keywordstyle    = {[3]\color{olive}},%
  stringstyle     = \color{teal},%
  commentstyle    = \itshape\color{magenta}}
\usepackage{floatrow} %% Putting captions above tables
\floatsetup[table]{capposition=top}
\begin{document}
\chapter*{Introduction}
introduction to problem + 
description of chapters
\addcontentsline{toc}{chapter}{Introduction}



%% We will define several mathematical sectioning commands.
\newtheorem{theorem}{Theorem}[section] %% The numbering of theorems
                               %% will be reset after each section.
\newtheorem{lemma}[theorem]{Lemma}         %% The numbering of lemmas
\newtheorem{corollary}[theorem]{Corollary} %% and corollaries will
                               %% share the counter with theorems.
\theoremstyle{definition}
\newtheorem{definition}{Definition}
\theoremstyle{remark}
\newtheorem*{remark}{Remark}

\counterwithin{definition}{chapter}

\chapter{Preliminaries and Definitions}
In this chapter, we firstly introduced notation and definitions used throughout this thesis. We used it to make a formal description of the problem of cylindrical algebraic decomposition. And finally, we describe and define a more specific version of this problem, which is a layered algebraic sub-decomposition.
\section{Polynomial definition and its characteristics}
Firstly we formulate the definition of the univariate polynomial, and it's elementary characteristics.
\begin{definition}{\textbf{Univariate polynomial}}
\newline
Univariate polynomial of degree $n$ over the ring $R$ is a function in the form:
\begin{align*}
p(x) =  \sum_{i=0}^n a_ix^i;\ \ \  a_i \in R, a_n \neq 0 
\end{align*}
The elements $a_0, a_1 .. a_n$ are called the coefficients of the polynomial $p$. Especially the coefficient $a_0$ is called constant coefficient and $a_n$ leading coefficient of the polynomial $p$.
\newline\newline
The set of all polynomial over ring $R$ is called the ring of the polynomials over $R$, and we represent it with symbol $R[x]$.
\end{definition}


After defining univariate polynomial, we describe some naming of some characteristics of the polynomials. The specific cases of the polynomials, which we use overall thesis,  are zero, constant and irreducible polynomial.  
\begin{definition}
Polynomial $f$ is called \textbf{zero polynomial} if all its coefficient equals to zero. All other polynomials are then called non-zero polynomial.
\end{definition}

\begin{definition}
Polynomials, which coefficient are all equal zero except the constant coefficient $a_0$ are called \textbf{constant polynomials}. All other polynomials are then called non-constant.
\end{definition}

\begin{definition}
Polynomial $p \in R[x]$ is called \textbf{irreducible} if $p$ is not constant and is not equal to product of two non-constant polynomials from ring $R[x]$
\end{definition}

\begin{definition}
\textbf{Root of polynomial $p$} is such a number $\alpha$ that satisfy the condition $p(\alpha) = 0$. Root $\alpha$ with multiplicity $k$ is called such a root of $p$ if exists polynomial $q(x)$ such that:
\begin{align*}
p(x) =  (x - \alpha)q(x)
\end{align*}
\end{definition}

We use also polynomials in two or more variables. Ring $R[x][y]$ consists of univariate polynomials in $y$ with coefficients in $R[x]$, but by collecting powers of $x$, we may as well consider its elements as univariate polynomials in $x$ with coefficients in $R[y]$. 

To reflect this symmetry, we use the notation for the ring in two variable $R[x,y]$, and more generally we can defined multivariate polynomial in $n$ variables over ring $R$ and denote the ring of polynomials in $n$ variables $R[x_1, x_2, \dots, x_n]$. We define multivariate polynomial in $n$ variables over ring $R$ gradually by defining concept of monomial and terms.

\begin{definition}{\textbf{Multivariate polynomial}}

\begin{itemize} 
  \item Every exponent vector $e  = (e_1, e_2, \dots, e_n) \in \mathbb{N}$ defines a monomial in $R[x_1, x_2 \dots, x_n]: x^e = x_1^{e_1} . x_2^{e_2}  \dots x_n^{e_n}$
  \item A term in polynomial $p \in R[x_1, x_2, . . . , x_n]$ is the product of a nonzero coefficient $r \in R \setminus \{0\}$ and some monomial.
  \item A multivariate polynomial $p \in R[x_1, x_2, . . . , x_n]$ is a finite sum of terms.
\end{itemize}

\end{definition}


We denote the degree and the leading coefficient of such a multivariate polynomial $p$ with respect to the variable $x_i$ by $deg_{x_i} p$ and $lc_{x_i}(p)$, respectively. 

The total \textbf{degree of a multivariate monomial} $x_1^{e_1}, x_2^{e_2}, \dots x_n^{e_n}$ equals to  $e_1 + e_2 \dots + e_n$, and the total degree of non-zero polynomial is the maximal total degree of its monomials.\cite{gathen_gerhard_2013} The degree of zero polynomial is defined as minus infinity.

\section{Cylindrical algebraic decomposition}
Firstly we need to define notation of a set invariance. 
\begin{definition}
Lets denote $E^k$ the Cartesian product $E \dots E, k \leq 1$, $S \subseteq E^k$ and $f$ be real valued function defined on $S$ then $f$ is invariant on $S$ or $S$ is f-invariant, when one of the following conditions holds:
\begin{itemize}
    \item $f(\alpha) > 0,\ \alpha \in S$ (f has positive sign on S) 
    \item $f(\alpha) = 0,\ \alpha \in S$ (f has zero sign on S)
    \item $f(\alpha) < 0,\ \alpha \in S$ (f has negative sign on S) 
\end{itemize}
\end{definition}

\begin{definition}
The \textbf{region} $R$ is a nonempty connected subset of $E^k$ and \textbf{decomposition} of $S \subseteq E^k$ is a finite collection of disjoint regions whose union is whole $S$.
\end{definition}

\begin{definition}
\textbf{Cylinder} over region $R$ written $Z(R)$ is a product $R x E$.
\newline
A \textbf{section} of $Z(R)$ is a set $S$ of points $[a, f(a)]$,where $a$ ranges over $R$, and $f$ is a continuous, real-valued function on $R$. In other words $S$ is graph of $f$ over region $R$
\end{definition}

\begin{definition}{\textbf{Cylindrical decomposition}}
\newline
Lets be $ f_1 < f_2 \dots < f_n \ n \geq 0$ continuous, real-valued functions
defined on $R$,these functions naturally determine decomposition of $Z(R)$ consisting of the following regions:
\begin{itemize}
    \item the $(f_i,f_{i+1})$-sectors of $Z(R)$ for $O \leq i leq n$ . where $f_o =  -\infty$ and $f_{k + 1} =  \infty$
    \item the $f_i$ sections of $Z (R)$ for $1 \leq i leq n$.
\end{itemize}
    We call such a decomposition a \textbf{stack over $R$} determined by functions $f_1, \dots f_n$
And  decomposition $D$ of $E^k$ is cylindrical if either of these conditions holds
\begin{itemize}
    \item  $k = 1$ and $D$ is a stack over $E_O$
    \item $k > 1$. and there is a cylindrical decomposition $D'$ of $E^{k-1}$ such that for each region $R$ of $D'$ some subset of $D$ is a stack over $R$. 
\end{itemize}
\end{definition}

\begin{definition}{\textbf{Algebraic decomposition}}
\newline
The \textbf{semi-algebraic set} is element of the class, which is defined as the smallest class of subsets of
$\mathbb{R}^{n}$ satisfying the following properties:
\begin{itemize}
    \item It contains all sets of the form $\{x\in \mathbb{R}^{n}:P(x)>0\},$ $P\in \mathbb{R}[X_{1}, \cdots, X_{n}]$ .
    \item It is stable under taking finite unions, finite intersections and complements.
\end{itemize}

A decomposition is algebraic if each of its regions is a semi-algebraic set.
\end{definition}

\begin{definition}
A \textbf{cylindrical algebraic decomposition} of $R^k$ is a decomposition which is
both cylindrical and algebraic.
\end{definition}

\section{Cylindrical algebraic sub-decomposition}
definition of subcad why subcad complexity
\chapter{Background and context}
The algorithm of cylindrical algebraic decomposition(CAD) was discovered by George E. Collins in 1973 and two years later published in his work \citetitle{10.1007/3-540-07407-4_17}. The paper describes the CAD method, which is much more efficient than the previous methods for a qualifier eliminations.\cite{10.1007/3-540-07407-4_17}. But since that it becomes a powerful technique in real algebraic geometry for studying semi-algebraic sets and in many other fields, which include robot motion planning\cite{LaValle:2006:PA:1213331},
epidemic modelling\cite{epidemicModeling} and theorem proving\cite{10.1007/978-3-642-32347-8_1}.

But because of the high complexity of CAD algorithm, it is often replaced by other heuristics methods, when the number of input polynomials is too high.

The problem is that CAD usually produces far more information than required to solve the underlying problem.\cite{Wilson2014}. We often want only a subset of a CAD sufficient to solve a given problem, which is called sub-CAD. And the primary focus of this thesis is to implement such changes in the method of basic Collins CAD algorithms, which enables us only to solve the sub-CAD problem. 

\section{State of art}
The state of art of the implementations of the CAD and the sub-CAD problem is that nowadays there is only a few available open source implementation of CAD and none of the sub-CAD problem.

Maybe first open source implementation which offers cad algorithm is QEPCAD
\footnote{
  See the documentation on \url{https://www.usna.edu/Users/cs/wcbrown/qepcad/B/QEPCAD.html}.
}. It is an interactive command-line program written in C language. The disadvantages of this open source application are that it does not implement the layered algebraic sub-decomposition and from information available in its web-pages, user documentation was last updated on 31 July 2002, and the latest update to the code of implementation available in log files is from 16 March 2012. On top of that, this implementation has no version control repository and few active users, so it would be tough to collaborate on this project.

Other available implementation of CAD algorithm offers Carl,
\footnote{
  See the documentation on: \url{https://smtrat.github.io/carl/classcarl_1_1CAD.html}
}
Which is an Open Source C++ Library for Computer Arithmetic and Logic. Although this implementation is available on the GitHub repository, it neither supports any implementations for the sub-CAD problem.
Only usable implementation of the sub-CAD are available in commercial tools like Maple 
\footnote{
See the Maple documentation on: \url{https://www.maplesoft.com/support/help/maple/view.aspx?path=RegularChains\%2FSemiAlgebraicSetTools\%2FCylindricalAlgebraicDecompose}.
}
or Wolfram Mathematica 
\footnote{
  See the Wolfram Mathematica documentation on \url{https://reference.wolfram.com/language/ref/CylindricalDecomposition.html}.
}.

After his research of the state of art of CAD and sub-CAD implementations, we deduce that there is a need for modern open source tool for the sub-CAD algorithm.

\chapter{Algorithm}
\section{Projection phase}
\section{Base phase}
In the base phase of our algorithm, we have to deal with the problem of computing the real roots of a univariate polynomial This problem  could be considered as the fundamental problem of computational algebra\parencite{yap2000fundamental} as well as in many applications ranging from computational geometry to quantifier elimination. 

We decided to solve this problem by implementing real root isolation with the arbitrary precision on the length of intervals. The problem of real root isolation consists of computing intervals with rational endpoints that contain exactly one real root of the polynomial.\parencite{10.1007/11841036_72}

Among many algorithms we decided to implement one based on Descartes's rule of signs together with certain Möbius transformation.

 \subsection{Requirements of Descartes's method}
 The Descartes method requirement for terminating is that input polynomial is square-free.\parencite{ganzha2005computer}
  \begin{definition}
 Given a polynomial $p$ in $R[x]$ where $R$ is a unique factorization domain, $p$ is said to be square-free if $p$ has no divisor (or factor) of multiplicity bigger than one.\parencite{Yun:1976:SDA:800205.806320}
\end{definition}

 Therefore we must ensure this condition before for Descartes's root isolation algorithm. That can be implemented by square-free factorization or polynomial factorization over rational numbers. We implemented both algorithms to see which would have better overall performance in the final implementation.

 The advantages of polynomial factorization are that it provides all the rational roots of a polynomial, which we do not have to find with Descartes's algorithm, but on the other hand, significant advantages of square-free factorization could be its time complexity. 
 
 In our implementation, we used Yun's algorithm from \citetitle{gathen_gerhard_2013} for square-free factorization and for polynomial factorization we used factorization modulo small prime followed by Hensel lifting and naive recombination. In the last chapter, we compare these two algorithms.
 
\subsection{Descartes's method for root isolation}
The advantages of this method are the simplicity of the implementation and a better theoretical upper bound for its computing time on integer polynomials against other techniques such as the Collins-Loos or Heindel's algorithms. 
 For a square-free integral polynomial of degree $n$, we obtain that time complexity of the algorithm belongs to  $O(n^6)$  \parencite{Collins:1976}
 
The basis for our algorithm is Descartes Rule of Signs
\begin{theorem}
[Descartes Rule of Signs]
  Let $p(x) = \sum_{i=0}^n b_ix^i [c, d](x)$ be a non-zero polynomial than $var(b)$ exceeds the number of zeroes of p in the open interval $(c, d)$ by even non-negative integer.\parencite{ganzha2005computer} Where $var(b)$ represents then number of sign changes of $p(x)$
  \end{theorem}
  
This theorem was first mentioned in Descartes's essay The Geometry in 1637 \cite{descartes2007geometry} and later stated and proved in exact form by Gauss in 1828 \cite{gauss1828}. Nevertheless it has two interesting consequences for our algorithm. Primarily if $var(b)$ on interval $ I = (c, d)$ of polynomial $p(x)$ is zero than interval $I$ does not contain any zero of  $p(x)$ and furthermore if $var(b) = 1 $, than $I$ is isolating interval for and contains exactly one root of $p(x)$.

In order to use the consequences in our algorithm we must to know how to transform a polynomial to arbitrary interval.
For this purpose we used Möbius transformation $x \gets \frac{ax + b}{x+1}$  which maps real positive line to interval $(a, b)$. \cite{DBLP:journals/corr/abs-1109-6279} Thus our function for transforming polynomial $p(x)$ is form: 
\begin{equation}
    f(p(x)) =  (x + 1)^n . p(\frac{ax + b}{x+1})
\end{equation}
The next pseudo code of algorithm $DescartesRootIsolation$ combines the  Descartes Rule of Signs theorem and transformation of the polynomial to the arbitrary interval $(a, b)$


\begin{algorithm}[H]
    \SetKwInOut{Input}{Input}
    \SetKwInOut{Output}{Output}
    \Input{
    Univariate polynomial, Lower bound, Upper bound
}
    \Output{Sorted list of all real roots of polynomial}
    
 result $\gets$ emptyList\\
 middleValue $\gets$ $\frac{(upperBound - lowerBound)}{2}$\\
 \If{NumberOfSigneChanges(polynomial) == 0}{
  \Return emptyList
 }
 \If{NumberOfSignChanges(polynomial) == 1 AND
 \newline
 $upperBound - lowerBound < \epsilon $}{
     \Return listOf(middleValue)
 }
 \If{middleValue is root of polynomial}{
    result.append(middleValue)
 }
 leftPoly $\gets$ toInterval(polynomial, lowerBound, middleValue)\\
 rightPoly $\gets$ toInterval(polynomial, middleValue, upperBound)\\
 \Return result $\cup$
 \newline
 DescartesRootIsolation(leftPoly, lowerBound, middleValue) $\cup$
 \newline
 DescartesRootIsolation(rightPoly, middleValue, upperBound)
 \caption{DescartesRootIsolation}
\end{algorithm}

\paragraph{
vytvorenie base cells and indexing
}
\section{Extension phase}

\chapter{Implementation}
\section{Rings library}
popis kniznice
co a ako implementuje
\section{Code structure}
insert class diagram
insert sequence diagram
\section{Testing}
Unit testy
Code coverage

\chapter{Experimental evaluation}
\
\section{Comparison to the available implementations}
\chapter{Conclusion}

  \printbibliography[heading=bibintoc] %% Print the bibliography.
\end{document}
